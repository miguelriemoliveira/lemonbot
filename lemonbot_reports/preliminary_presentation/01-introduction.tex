\section{Introdução}

\begin{frame}{Overview}
	Reconstrução de espaços utilizando tecnologia Lidar			
\end{frame}

\subsection{Tecnologias}

\begin{frame}{Visão Estéreo}
			
	\begin{itemize}
		\item Mecanismo similar ao funcionamento estéreo do olho humano.
		\item Distância é calculada pelo cálculo da disparidade entre duas imagens. Esta disparidade é depois transformada numa nuvem de pontos.
		\item Não contem informação dimensional absoluta, apenas relativa.
	\end{itemize}
	
\end{frame}

\begin{frame}{RGBD}
	
	\begin{itemize}
		\item Utiliza um sensor destinado à captura da informação de depth. Usualmente é usado uma câmara IR que capta padrões desenhados por um laser.
		\item Resulta numa imagem com informação de cor (RGB) e distância (D).
		\item Esta tecnologia não é muito precisa.
	\end{itemize}
	
\end{frame}

\begin{frame}{Lidar}
	
	...
																
\end{frame}

\begin{frame}{Roadmap}
	    
	\begin{enumerate}
		        
		\item Aquisição da nuvem de pontos 3d
		      
		\item Calibração extrínseca da Câmara-PTU (hand2eye) e Câmara-Laser (radlocc).
		      
		\item Triangulação da nuvem de pontos.
		              
		\item Registo das cores à nuvem de pontos.
		      \begin{itemize}
		      	\item Atribuição da cor a cada vértice (simples).
		      	\item Atribuição da textura à malha (difícil).
		      \end{itemize}
              
        \item Uniformização da cor entre malhas. Variação é causada por variações nas condições da captura (tempo de exposição/color balance/luminosidade).

	\end{enumerate}
	
\end{frame}